\documentclass[12pt]{article}
 \usepackage[margin=1in]{geometry} 
\usepackage{amsmath,amsthm,amssymb,amsfonts}
\usepackage{lipsum}
 
\newcommand{\N}{\mathbb{N}}
\newcommand{\Z}{\mathbb{Z}}
 
\newenvironment{problem}[2][Problem]{\begin{trivlist}
\item[\hskip \labelsep {\bfseries #1}\hskip \labelsep {\bfseries #2.}]}{\end{trivlist}}
%If you want to title your bold things something different just make another thing exactly like this but replace "problem" with the name of the thing you want, like theorem or lemma or whatever
 
\begin{document}

%\renewcommand{\qedsymbol}{\filledbox}
%Good resources for looking up how to do stuff:
%Binary operators: http://www.access2science.com/latex/Binary.html
%General help: http://en.wikibooks.org/wiki/LaTeX/Mathematics
%Or just google stuff
 
\title{489 - Assignment 1}
\author{Onur Tirtir\\2099380}
\maketitle

\section{CBC Mode}
\par
In CBC mode, each block of plain-text is first $xor$'ed by previous cipher-text block 
retrieved by encryption of previous plain-text block and then encrypted by \textit{key}.
This brings us the \textit{diffusion} property.
\par
Even if we have a trivial plain-image like "original.bmp", we cannot deduce any
useful information in first glance to the cipher-image as it seems much like noise.
Noisiness comes directly from "cipher block chaining".

\section{ECB Mode}
\par
In ECB mode, each block is encrypted independently from the other blocks.
ECB mode does not utilize any feed-backs from previous blocks unlike some other
modes (like CBC Mode). Regardless of its position, a plain-text block 
is always encrypted to same cipher-text block. Hence ECB Mode cannot
satisfy \textit{diffusion} property.
\par
Near pixels have same RGB values if the input image is not that complex
due to the reasons explained above.
Hence we may deduce some or most parts of underlying plain-image, its rough shape and 
even the objects in image depending on the 
color-wise complexity of given plain-image.

\end{document}

