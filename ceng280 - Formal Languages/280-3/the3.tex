\documentclass[12pt]{article}
\usepackage[utf8]{inputenc}
\usepackage{float}
\usepackage{amsmath}


\usepackage[hmargin=3cm,vmargin=6.0cm]{geometry}
%\topmargin=0cm
\topmargin=-2cm
\addtolength{\textheight}{6.5cm}
\addtolength{\textwidth}{2.0cm}
%\setlength{\leftmargin}{-5cm}
\setlength{\oddsidemargin}{0.0cm}
\setlength{\evensidemargin}{0.0cm}

%misc libraries goes here
\usepackage{tikz}
\usetikzlibrary{automata,positioning}

\newtheorem{thm}{Proof}

\begin{document}

%TODO %TODO %TODO %TODO %TODO %TODO %TODO %TODO %TODO %TODO %TODO %TODO %TODO %TODO %TODO
%    1-c'yi yap  ok
%    
%    1-b'yi anla ok
%TODO %TODO %TODO %TODO %TODO %TODO %TODO %TODO %TODO %TODO %TODO %TODO %TODO %TODO %TODO

\section*{Student Information} 
%Write your full name and id number between the colon and newline
%Put one empty space character after colon and before newline
Full Name :  Onur Can TIRTIR\\
Id Number :  2099380\\

% Write your answers below the section tags
\section*{Answer 1}

\subsection*{a.}

For any language $L_k \in \Sigma^*$, $\overline{L_k} = L_1 \cup L_2 \cup \dots \cup L_{k-1
} \cup L_{k+1} \cup L_{k+2} \dots \cup L_{n}$, which is recursively enumerable by
following:

    \begin{thm}
    Say we have two recursively enumerable languages $L_1$ and $L_2$ such that they can 
be recognized by the Turing Machines $M_1$ and $M_2$. If we construct a new Turing Machine
 $M$ which recognizes their union, then we can say that whenever either of $M_1$ and $M_2$
 recognizes $w$, $M$ also accepts $w$ and when both rejects $w$, $M$ also rejects $w$, 
so these satisfy $M = M_1 \cup M_2$. It is easy to see that $M$ runs as indicated above:\\
    For any $w$, run $M_1$ and $M_2$ parallel. If either TM accepts $w$, which is done in
finitely many steps, then $M$ accepts $w$. If both $M_1$ and $M_2$ rejects $w$ then both 
will loop infinitely, which makes $M$ also loop. So $M$ never accepts $w$ as $M_1$ and 
$M_2$ do.
    \end{thm}

So both $L_k$ and $\overline L_k$ are recursively enumerable then by the theorem 5.7.1 
in Chapter 5.7, \textit{a language is recursive iff both it and its complement are 
recursively enumerable}, $L_k$ is recursive.

\subsection*{b.}

Since $L_1$ and $\overline L_1$ are recursively enumerable then by the theorem 5.7.1 
in Chapter 5.7, $L_1$ is recursive. $L_2 \setminus L_1 = L_2 \cap \overline L_1$
. As stated by the theorem 4.2.2 in Chapter 4.2, recursive languages are closed under 
complement, then $\overline L_1$ is also recursive. Now prove that recursive languages 
are also closed under intersection.

    \begin{thm}
    Say we have two recursive languages $L_1$ and $L_2$ such that they can be recognized
 by the Turing Machines $M_1$ and $M_2$. If we construct a new Turing Machine $M$ which
 recognizes their intersection, then we can say that whenever both of $M_1$ and $M_2$
 accepts $w$, $M$ also accepts $w$ and whenever either of them rejects $w$, $M$ also 
rejects $w$, so these satisfy $M = M_1 \cap M_2$. It is easy to see that $M$ runs as 
indicated above:\\
    For any $w$, first run $M_1$. If $M_1$ accepts $w$, then run $M_2$. If $M_2$ acceps 
$w$, then $w$ is recognized by $M$. If $M_1$ does reject $w$, or $M_1$ accepts but $M_2$ 
rejects $w$, then $w$ cannot be recognized by $M$. Note that each rejection and
acceptance is done in finitely many steps.
    \end{thm}

Hence $L_2 \cap \overline L_1$ is also recursive by the closure property indicated above.
\\ \\

If $L_2$ were recursively enumerable, $L_2 \cap \overline L_1$ will also be recursively
 enumerable. Here is the reasons:\\
1) $\overline L_1$ were recursive. This means it is actually recursively enumerable.\\
2) $L_2$ is recursively enumerable.\\
3) Since recursively enumerable languages are closed under intersection, then $L_2 \cap
 \overline L_1$ is also. Now prove recursively enumerable languages are closed under
 intersection.

    \begin{thm}
    Say we have two recursively enumerable languages $L_1$ and $L_2$ such that they can 
be recognized by the Turing Machines $M_1$ and $M_2$. If we construct a new Turing
 Machine $M$ which recognizes their intersection, then we can say that whenever both of 
$M_1$ and $M_2$ recognizes $w$, $M$ also accepts $w$ and when either of them rejects 
$w$, $M$ also rejects $w$, so these satisfy $M = M_1 \cap M_2$. It is easy to see that 
$M$ runs as indicated above:\\
    For any $w$, run $M_1$ and $M_2$ sequentially. If either TM accepts $w$, which is 
done in finitely many steps, then $M$ accepts $w$. If either of $M_1$ and $M_2$ rejects 
$w$ then $M$ will loop infinitely. So $M$ never accepts.
    \end{thm}

\subsection*{c.}

    First give a little explanation recursive languages are closed under concatanetion.
    \begin{thm}
    Say we have two recursive languages $L_1$ and $L_2$ such that they can be recognized
 by the Turing Machines $M_1$ and $M_2$. As illustrated in 4-c, TMs can be concataneted.
 Hence TMs are closed under concatanetion.
    \end{thm}
    
    $L_N = L_1 \setminus (L_2L_3) = L_1 \cap \overline{L_2L_3}$, which is also a 
recursive language by the closure properties given in this and above parts.\\
For any function $f$'s computablity means that there exists a $M$ such that $M$ computes
 $f$ for all $w \in \Sigma_0^*$, i.e $M(w) = f(w)$. That is to say, for any string from 
$\Sigma_0^*$, $M$ \textbf{eventually halts on input w}. The bold words says, by the 
definition of decidibility, $M$ is decidable. Hence $f$ is also decidable. If $f$ is a 
bijective mapping from some strings to another strings, then all the things we just said
 for such a function $f$ are all valid for its inverse because there is no reason 
for us not to be able to compute the outcome of $f^{-1}$ properly as f was a bijection.
\\ \\
Now take the specific function $f^{-1}$ given in the question at hand. 
$L_N$ is recursive. That means there is a $M$ that decides it. Since $f^{-1}$ is 
decidable, then we can say that the TM $M$ computing $f^{-1}$ can decide on $L_n$.\\
So $f^{-1}(L_N)$ is Turing-decidable.

\section*{Answer 2}

\subsection*{a.}

$G = \{V, \Sigma, R, S\}$\\
$V = \{S, L, D, R, a\}$\\
$\Sigma = \{a\}$
    \begin{flalign*}
    & R = \{ S  \rightarrow LaD, &\\
    &        D  \rightarrow RD , &\\
    &        D  \rightarrow e  , &\\
    &        aR \rightarrow Raa, &\\
    &        LR \rightarrow L  , &\\
    &        L  \rightarrow e \} &\\
    \end{flalign*}

\subsection*{b.}

$G = \{V, \Sigma, R, S^{'}\}$\\
$V = \{S^{'}, S, 1, 2, x, y, X, Y, L, E_x, E_y, E_{1,2}\}$\\
$\Sigma = \{1, 2, x, y\}$
    \begin{flalign*}
    & R = \{ S^{'}    \rightarrow LS,       &\\ %Start%
    &        S        \rightarrow E_y,      &\\ %Generate%
    &        S        \rightarrow 1YS,      &\\ %
    &        S        \rightarrow 2XS,      &\\ %Generate%
    &        X1       \rightarrow 1X,       &\\ %Sort% 
    &        Y1       \rightarrow 1Y,       &\\ %
    &        X2       \rightarrow 2X,       &\\ %
    &        Y2       \rightarrow 2Y,       &\\ %
    &        YX       \rightarrow XY,       &\\ %Sort%
    &        YE_y     \rightarrow E_yy,     &\\ %RightToLeftCorrect%
    &        YE_y     \rightarrow E_xy,     &\\ %
    &        XE_x     \rightarrow E_xx,     &\\ %
    &        XE_x     \rightarrow E_{1,2}x, &\\ %
    &        1E_{1,2} \rightarrow E_{1,2}1, &\\ %
    &        2E_{1,2} \rightarrow E_{1,2}2, &\\ %RightToLeftCorrect%
    &        LE_{1,2} \rightarrow e,        &\\ %End
    &        E_y      \rightarrow E_x,      &\\ %Jump
    &        E_y      \rightarrow E_{1,2},  &\\ %
    &        E_x      \rightarrow E_{1,2} \}&\\ %Jump
    \end{flalign*}

\section*{Answer 3}

Note: If there is no go to condition, directives given below will be followed sequentially

\subsection*{a.}

0-) If the input is not in the format $a^pb^rc^tx^ky^lz^m$,
 where $p \geq 1, r \geq 1, t \geq 1, k \geq 1, l \geq 1, m \geq 1,$ then reject,
 else move head to leftmost symbol

A1-) Convert $a$ to $A$.

A2-) Move head to right to find leftmost $b$ in tape. If there is no $b$, then reject.

A3-) Convert $b$ to $B$. 

A4-) Move head right to find leftmost $c$ in tape. If there is no $c$, then reject.

A5-) Convert $c$ to $C$

A6-) Try to find leftmost $b$ by moving head to left. If found any then go to step A3.

A7-) Move head left sequentially and convert all $B$s to $b$ while moving left.

A8-) Try to find leftmost $a$ by moving head to left. If found one $a$ then go to step A1.

A9-) Try to find leftmost $c$ by moving head to right. If found then reject.

B1-) Find leftmost $x$ by moving head to right.

B2-) Convert $x$ to $X$.

B3-) Try to find leftmost $y$ by moving head to right. If there is no $y$ then reject.

B4-) Convert $y$ to $Y$.

B5-) Try to find leftmost $z$ by moving head to right. If there is no $z$ then reject.

B6-) Convert $z$ to $Z$.

B7-) Try to find leftmost $x$ by moving head to left. If found then go to step B2.

B8-) Try to find leftmost $y$ by moving head to right. If found then reject.

B9-) Try to find leftmost $z$ by moving head to right. If found then reject.

H-) Accept.

\subsection*{b.}

I will use three tapes, tape1 will keep the input and, tape2 and tape3 will be used to keep the 
copy of the strings to be compared. In this subsection, head implies the head of tape1 
unless otherwise is indicated.

0-) Find first dollar symbol by moving head to right until finding and mark it as $X$.

1-) Move back to first symbol in tape and then check if there is any $Y$ symbol by moving 
to right.

2-) If $Y$ is found then go to step A1, else go to step B1.

A1-) Convert current $Y$ to dollar. Try to find next dollar symbol after $Y$ by moving 
head to right. If there is then mark it $Y$ and go to step Comp1.

A2-) Move back to first symbol in tape and find $X$ symbol by moving right. Make it 
dollar.

A3-) Try to find next dollar symbol after current dollar symbol by moving head to right.

A4-) If dollar is found then make it $X$ and go to step 1, else accept.

B1-) Move back to first symbol in tape and try to find the leftmost dollar symbol by 
moving head to right.

B2-) If dollar is found then make it $Y$ and go to step Comp1, else accept.

Comp1-) Copy the string coming after $X$ to TM1 excluding ending dollar symbol and 
copy the string coming after $Y$ to TM2 excluding ending dollar symbol.

Comp2-) Move heads of TM1 and TM2 in paralled until one of the heads reads $\sqcup$. 
If both heads read same symbol except $\sqcup$ then move both heads just one right.
Else if they read different symbols then go to step 1. Else if both reads $\sqcup$ then 
reject.

\section*{Answer 4}

\subsection*{a.}

$M      = \{K, \Sigma, \delta, s, H\}$ where:\\
$K      = \{s, q_0, q_1, q_2, q_3, q_4, q_5, h\}$,\\
$\Sigma = \{\sqcup, 1, x, y, \triangleright   \}$,\\
$H      = \{h                                 \}$, and $\delta$ is defined such that:\\ \\
$\delta(s, \sqcup) = (s, \rightarrow)$\\
$\delta(s, 1) = (q_0, \rightarrow)$\\
$\delta(q_0, 1) = (q_0, \rightarrow)$\\
$\delta(q_0, \sqcup) = (q_1, y)$\\
$\delta(q_1, y) = (q_1, \leftarrow)$\\
$\delta(q_1, 1) = (q_1, \leftarrow)$\\
$\delta(q_1, \sqcup) = (q_2, \rightarrow)$\\
$\delta(q_1, x) = (q_2, \rightarrow)$\\
$\delta(q_2, 1) = (q_3, x)$\\
$\delta(q_3, x) = (q_3, \rightarrow)$\\
$\delta(q_3, y) = (q_3, \rightarrow)$\\
$\delta(q_3, 1) = (q_3, \rightarrow)$\\
$\delta(q_3, \sqcup) = (q_1, 1)$\\
$\delta(q_2, y) = (q_5, 1)$\\
$\delta(q_5, 1) = (q_4, \leftarrow)$\\
$\delta(q_4, x) = (q_5, 1)$\\
$\delta(q_4, \sqcup) = (h, \sqcup)$\\

Also note that $\forall q \in K - H,\ \delta(q, \triangleright) = (p, \rightarrow).$ \\

Our TM's behaviour will be like that:\\
1-) Find first blank symbol after string and put a $y$ here to mark.\\
2-) While there exists at least one $1$ before $y$, convert leftmost $1$ to $x$ and put
 another $1$ after symbol $y$.\\
3-) Convert $y$ to $1$.\\
4-) While there exists at least one $x$ before $y$, convert rightmost $x$ to $1$ and
 move left.\\
5-) Halt when arrived to blank symbol right after $\triangleright$.

\subsection*{b.}

$M      = \{K, \Sigma, \delta, s, H\}$ where:\\
$K      = \{s, q_0, q_1, q_2, q_3, q_4, q_5, q_6, q_7, q_8, h\}$,\\
$\Sigma = \{\sqcup, 1, x, \triangleright                     \}$,\\
$H      = \{h                                                \}$, and $\delta$ is defined 
such that:\\ \\
$\delta(s, \sqcup) = (s, \rightarrow)$\\
$\delta(s, 1) = (q_0, 1)$\\
$\delta(q_0, 1) = (q_0, x)$\\
$\delta(q_0, x) = (q_1, \rightarrow)$\\
$\delta(q_1, 1) = (q_1, \rightarrow)$\\
$\delta(q_1, \sqcup) = (q_2, \leftarrow)$\\
$\delta(q_2, 1) = (q_3, \sqcup)$\\
$\delta(q_3, \sqcup) = (q_3, \leftarrow)$\\
$\delta(q_3, 1) = (q_4, \sqcup)$\\
$\delta(q_4, \sqcup) = (q_4, \leftarrow)$\\
$\delta(q_4, 1) = (q_5, \sqcup)$\\
$\delta(q_5, \sqcup) = (q_5, \leftarrow)$\\
$\delta(q_5, 1) = (q_6, \sqcup)$\\
$\delta(q_6, \sqcup) = (q_6, \leftarrow)$\\
$\delta(q_6, 1) = (q_7, 1)$\\
$\delta(q_7, 1) = (q_7, \leftarrow)$\\
$\delta(q_7, x) = (q_0, \rightarrow)$\\
$\delta(q_2, x) = (q_8, x)$\\
$\delta(q_3, x) = (q_8, x)$\\
$\delta(q_4, x) = (q_8, x)$\\
$\delta(q_5, x) = (q_8, x)$\\
$\delta(q_6, x) = (q_8, x)$\\
$\delta(q_8, 1) = (q_8, \leftarrow)$\\
$\delta(q_8, x) = (q_8, 1)$\\
$\delta(q_8, \sqcup) = (h, \sqcup)$\\

Also note that $\forall q \in K - H,\ \delta(q, \triangleright) = (p, \rightarrow).$ \\

Our TM's behaviour will be like that:\\
1-) Find leftmost $1$ and mark it as $x$.\\
2-) Start from the rightmost $1$ in tape and try to delete(convert to $\sqcup$) $4$ 
rightmost $1$s by traversing the tape starting from the rightmost $1$ and goint to left.\\
3-) If you succeded and have still $1$s in tape(checking that the symbol left to 
current head position is $1$ or $x$) then go to step $1$. Else you have $0,\ 1,\ 2$ or 
$3$ $1$s and you delete them.\\
4-) Convert all the $x$'s in tape $1$ by traversing the tape right to left.\\

\subsection*{c.}

Rename $M$ for TM in part a as $M_1$, $K$ as $K_1$, $\delta$ as $\delta_1$ and $H$ as 
$H_1$. Add subscript $_1$ for all $q \in K$. Redefine all the definitions of $\delta$ 
for $\delta_1$.\\

Rename $M$ for TM in part a as $M_2$, $K$ as $K_2$, $\delta$ as $\delta_2$ and $H$ as 
$H_2$. Add subscript $_2$ for all $q \in K$. Redefine all the definitions of $\delta$ 
for $\delta_2$. Also append the $y$ symbol to alphabet of $M_2 $ to make it equivalent 
to alphabet of TM in part a, so as to make it consistent for our new TM $M_3$, which will
be the answer of this part.\\

Then:\\

$\Sigma = \{\sqcup, 1, x, y, \triangleright\}$,\\

$M_1  = \{K_1, \Sigma, \delta_1, s_1, H_1\}$ where:\\
$K_1  = \{s_1, q_{10}, q_{11}, q_{12}, q_{13}, q_{14}, q_{15}, h_1\}$,\\
$H_1  = \{h_1 \}$, and $\delta_1$ is defined such that:\\ \\
$\delta_1(s_1, \sqcup) = (s_1, \rightarrow)$\\
$\delta_1(s_1, 1) = (q_{10}, \rightarrow)$\\
$\delta_1(q_{10}, 1) = (q_{10}, \rightarrow)$\\
$\delta_1(q_{10}, \sqcup) = (q_{11}, y)$\\
$\delta_1(q_{11}, y) = (q_{11}, \leftarrow)$\\
$\delta_1(q_{11}, 1) = (q_{11}, \leftarrow)$\\
$\delta_1(q_{11}, \sqcup) = (q_{12}, \rightarrow)$\\
$\delta_1(q_{11}, x) = (q_{12}, \rightarrow)$\\
$\delta_1(q_{12}, 1) = (q_{13}, x)$\\
$\delta_1(q_{13}, x) = (q_{13}, \rightarrow)$\\
$\delta_1(q_{13}, y) = (q_{13}, \rightarrow)$\\
$\delta_1(q_{13}, 1) = (q_{13}, \rightarrow)$\\
$\delta_1(q_{13}, \sqcup) = (q_{11}, 1)$\\
$\delta_1(q_{12}, y) = (q_{15}, 1)$\\
$\delta_1(q_{15}, 1) = (q_{14}, \leftarrow)$\\
$\delta_1(q_{14}, x) = (q_{15}, 1)$\\
$\delta_1(q_{14}, \sqcup) = (h_1, \sqcup)$\\

Also note that $\forall q \in K_1 - H_1,\ \delta_1(q, \triangleright) = (p, \rightarrow).$
 \\

$M_2 = \{K_2, \Sigma, \delta_2, s_2, H_2\}$ where:\\
$K_2 = \{s_2, q_{20}, q_{21}, q_{22}, q_{23}, q_{24}, 
 q_{25}, q_{26}, q_{27}, q_{28}, h_2\}$,\\
$H_2 = \{h_2\}$, and $\delta_2$ is defined 
such that:\\ \\
$\delta_2(s_2, \sqcup) = (s_2, \rightarrow)$\\
$\delta_2(s_2, 1) = (q_{20}, 1)$\\
$\delta_2(q_{20}, 1) = (q_{20}, x)$\\
$\delta_2(q_{20}, x) = (q_{21}, \rightarrow)$\\
$\delta_2(q_{21}, 1) = (q_{21}, \rightarrow)$\\
$\delta_2(q_{21}, \sqcup) = (q_{22}, \leftarrow)$\\
$\delta_2(q_{22}, 1) = (q_{23}, \sqcup)$\\
$\delta_2(q_{23}, \sqcup) = (q_{23}, \leftarrow)$\\
$\delta_2(q_{23}, 1) = (q_{24}, \sqcup)$\\
$\delta_2(q_{24}, \sqcup) = (q_{24}, \leftarrow)$\\
$\delta_2(q_{24}, 1) = (q_{25}, \sqcup)$\\
$\delta_2(q_{25}, \sqcup) = (q_{25}, \leftarrow)$\\
$\delta_2(q_{25}, 1) = (q_{26}, \sqcup)$\\
$\delta_2(q_{26}, \sqcup) = (q_{26}, \leftarrow)$\\
$\delta_2(q_{26}, 1) = (q_{27}, 1)$\\
$\delta_2(q_{27}, 1) = (q_{27}, \leftarrow)$\\
$\delta_2(q_{27}, x) = (q_{20}, \rightarrow)$\\
$\delta_2(q_{22}, x) = (q_{28}, x)$\\
$\delta_2(q_{23}, x) = (q_{28}, x)$\\
$\delta_2(q_{24}, x) = (q_{28}, x)$\\
$\delta_2(q_{25}, x) = (q_{28}, x)$\\
$\delta_2(q_{26}, x) = (q_{28}, x)$\\
$\delta_2(q_{28}, 1) = (q_{28}, \leftarrow)$\\
$\delta_2(q_{28}, x) = (q_{28}, 1)$\\
$\delta_2(q_{28}, \sqcup) = (h_2, \sqcup)$\\

Also note that $\forall q \in K_2 - H_2,\ \delta_2(q, \triangleright) = (p, \rightarrow).$
\\ \\
And finally our answer is:\\

$M_3 = \{K_1 \cup K_2, \Sigma, \delta_3, s_2, H_1 \}$ and define $\delta_3$ such that:\\

$\forall \sigma \in \Sigma$:\\
1-) $If\ q \in K_1 - H_1,\ then\ \delta_3(q, \sigma) = \delta_1(q, \sigma)$\\
2-) $If\ q \in K_2 - H_2,\ then\ \delta_3(q, \sigma) = \delta_2(q, \sigma)$\\
3-) $If\ q \in H_2,\ then\ \delta_3(q, \sqcup) = (s_1, \sqcup)$.\\

Our TM $M_3$ will work such that: first $M_2$ runs and halts in the blank symbol just 
right of $\triangleright$, then $M_1$ runs and it also halts in that blank symbol and 
hence computation is done.

\section*{Answer 5}

\subsection*{a.}

Say we have a Turing Machine $M_1$, counting the number of states of $M$ ,say $s$, and 
simulating $M$ in a tape indicated in initial configuration up to $s$ moves. $M_1$ will
accept if $M$'s simualation writes a non-blank symbol. We also know if $M \notin L$ then
$M \notin L(M_1)$. For another case, assume $M \in L$ and $M$ will write first non-blank
symbol after $s'$ transitions and these transitions should definitely be startting in 
$s'$ states, In the opposite key, the mean $s'$ will contradict. So we can conclude that
$s' < s$ and $M'$ will accept. Thus $L$ is decidable. Note that this does not conflicts
with halting problem.

\subsection*{b.}

A decider, by definiton, should accept the strings in the language and rejects the 
strings not in the language and halt for both cases. Say we have a language $L$ which 
does not accept any string $w \in \Sigma_0^*$. So we can say that our TM $M$ is just 
looping whenever an input $w$ s.t $w \in \Sigma_0^*$ is provided. But $L$ is decidable 
since we can just define another TM $M'$ such that it, again, does not accept any 
possible input $w$,however, this property is not provided by $infinite\ looping$, so 
$M'$ decides $L$ and hence $L$ is decidable.

\subsection*{c.}

We do not know if $D_1$ accepts or rejects $w \notin L$. Also we do not know if $D_2$ 
accepts or rejects $w \in L$. Then there may exists such ambigious situations:\\ \\

For example, as we stated above, $D_2$ may reject the strings $w \in L$, but $D_1$ will 
certainly accept these strings. So we do not know these kind of words are in the language 
or not. So some of these language $L$s may not be decidable.

\section*{Answer 6}

\subsection*{a.}

In part b, we easily constructed a deterministic PDA since we know from which set $w$
came by the help of the symbols $x$ and $y$. However in part a, the absence of these 
strings may create some problems:\\
    For instance, to recognize a string like $a^nb^{2n}$, the rule is easy: for every $a$
push $aa$ to stack then pop $a$ for every $b$ in $w$. Since we do not know from 
which set $w$ came, it is a must to include this rule in $\Delta$. However, in case of 
a string $a^nb^{n+m}c^m$, where $m \neq n$, if we are to apply the same rule, then some
 $a$ will be left in stack. Another example is that if we use the rules for the strings 
from first set, then we fail to accept the strings from second set because there is no 
$c$ in these strings. In both examples, it was a must to use the rules indicated to 
construct our TM properly but we could not.\\
Then only solution is to put both transitions to $\Delta$ and benefit from backtracking, 
which makes our automaton a nondeterministic PDA. Then the language given in this part 
is not a deterministic CFG. 

\subsection*{b.}

We can write a deterministic pushdown automaton for $L$ such that:\\
$M = (K, \{a, b, c, x, y\}, \{a, b, c\}, \Delta, s, f)$ where
    \begin{flalign*}
    & \Delta = \{                            &\\
    &            ((s, x, e),   (q_0, x)),    &\\
    &            ((q_0, a, e), (q_1, a)),    &\\
    &            ((q_1, a, e), (q_1, a)),    &\\
    &            ((q_1, b, a), (q_1, e)),    &\\
    &            ((q_1, e, x), (q_2, x)),    &\\
    &            ((q_2, b, e), (q_2, b)),    &\\
    &            ((q_2, c, b), (q_2, e)),    &\\
    &            ((q_2, e, x), (f, e)),      &\\
    &            ((s, y, e),   (q_3, y)),    &\\
    &            ((q_3, a, e), (q_4, aa)),   &\\
    &            ((q_4, a, e), (q_4, aa)),   &\\
    &            ((q_4, b, a), (q_4, e)),    &\\
    &            ((q_4, e, y), (f, e))       &\\
    & \}                                     &\\
    \end{flalign*}

Since there is no two distinct compatible transitions in $M$, $M$ is deterministic. 
Then $L = L(M)$ is a CFG.

\end{document}

​

