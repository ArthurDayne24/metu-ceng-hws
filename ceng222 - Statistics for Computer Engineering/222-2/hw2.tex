\documentclass[12pt]{article}
\usepackage[utf8]{inputenc}
\usepackage{float}
\usepackage{amsmath}


\usepackage[hmargin=3cm,vmargin=6.0cm]{geometry}
%\topmargin=0cm
\topmargin=-2cm
\addtolength{\textheight}{6.5cm}
\addtolength{\textwidth}{2.0cm}
%\setlength{\leftmargin}{-5cm}
\setlength{\oddsidemargin}{0.0cm}
\setlength{\evensidemargin}{0.0cm}

\newcommand{\HRule}{\rule{\linewidth}{1mm}}

%misc libraries goes here
\usepackage{tikz}
\usetikzlibrary{automata,positioning}

\begin{document}

\noindent
\HRule \\[3mm]
\begin{flushright}

                                         \LARGE \textbf{CENG 222}  \\[4mm]
                                         \Large Statistical Methods for Computer Engineering \\[4mm]
                                        \normalsize      Spring '2016-2017 \\
                                           \Large   Assignment 2 \\
					\normalsize Deadline: March 26, 23:59 \\
					\normalsize Submission: via COW
\end{flushright}
\HRule

\section*{Student Information } 
%Write your full name and id number between the colon and newline
%Put one empty space character after colon and before newline
Full Name :  Onur Can TIRTIR\\
Id Number :  2099380\\

% Write your answers below the section tags
\section*{Answer 3.15}

\subsection*{a)}

Since we want to find the probability of at least one hardware failure, we need to subtract the probability that $0$ hardware failure from $1$.
Then the answer is $1-0.52=0.48$.

\subsection*{b)}

We need to find one contradiction to show that they are independent. \\
We can find $P(y=0)=\sum_{x=0}^2 P(x, y=0)=0.76$, $P(x=0)=\sum_{y=0}^2 P(x=0, y)=0.72$.\\
If these $P(x=0)$ and $P(y=0)$ are independent then $P(x=0)P(y=0)=P(x=0, y=0)$. But $0.5472 \neq 0.52$. Hence they are dependent.

\section*{Answer 3.32}

X is to be the number of crashed computers. We want to compute the number of \textit{successes(a\ computer\ crashed)} within 4000 computers, which are our trials$(n=4000)$. Here the probability of success is $p=1/800$. Hence $\lambda=np=5$. Since $n$ is large and $p$ is small then we have a binomial distribution.

\subsection*{a)}
$P(x<10)=F(9)=0.968$ from the table A3 in book.

\subsection*{b)}
$P(x=10)=F(10)-F(9)=0.986-0.968=0.018$ from the table A3 in book.

\section*{Answer 3.35}
Let $X$ is to be the number of traffic accidents occured yesterday and $T$ be the event(thunderstorm).\\
By Bayes Rule:
\begin{align*}
P(T | X=7)  & =\dfrac{P(X=7 | T)P(T)}{P(X=7)} \\
            & =\dfrac{P(X=7 | T)P(T)}{P(X=7 | T)P(T)+P(X=7 | \overline{T})P(\overline{T})} \\
			& =\dfrac{ (F_{\lambda=10}(7)-F_{\lambda=10}(6) )0.6}{ (F_{\lambda=10}(7)-F_{\lambda=10}(6) ) 0.6 + (F_{\lambda=4}(7)-F_{\lambda=4}(6) )  0.4} \\
		    & =\dfrac{(0.220-0.130)0.6}{(0.220-0.130)0.6+(0.949-0.889) 0.4} \\
			& = 0.6923076
\end{align*} 

\section*{Answer 4.4}

\subsection*{a)}
\begin{align*}
\int_{-\infty}^{+\infty}K-\dfrac{x}{50} &= \int_{0}^{10}K-\dfrac{x}{50}+0 &&  \text{Since in other areas f(x)=0} \\
& = (Kx-\dfrac{x^2}{100} )\vert_0^{10} \\ & = (10K-\dfrac{100}{100}) - 0 \\ & = 10K - 1 = 1
\end{align*}

Hence $10K=2$ then $K=0.2$.

\subsection*{b)}
\begin{align*}
\int_{0}^{5}0.2-\dfrac{x}{50} & = (0.2x - \dfrac{x^2}{100} ) \vert_0^5 \\
							  & = (1 - \dfrac{5^2}{100}) - 0 \\
							  & = (1 - \dfrac{1}{4}) \\
							  & = \dfrac{3}{4} \\
\end{align*}

\subsection*{c)}
$E(X)=\int_{-\infty}^{+\infty}xf(x)$
\begin{align*}
\int_{-\infty}^{+\infty}x(0.2-\dfrac{x}{50}) & = \int_{-\infty}^{+\infty}(0.2x-x\dfrac{x}{50}) \\											  & = \int_{0}^{10}(0.2x-x\dfrac{x}{50}) &&  \text{Since in other areas f(x)=0} \\ 				  
											 & = (0.1x^2-\dfrac{x^3}{150}) \vert_0^{10} \\ 
											 & = (0.1*100-\dfrac{100}{15}) - 0 \\
											 & = (10-\dfrac{20}{3}) \\
											 & = \dfrac{10}{3} \\
\end{align*}

\section*{Answer 4.10}

$W$ is to be the event that the orders is still not ready, in $30$th minute. That means $W$ is the event that the order given takes more than $30$ minutes. Let $S_1$ be the event that the order is taken by the first specialist. Let $S_2$ be the event that the order is taken by the second specialist. Note that $S_1$ and $S_2$ are disjoint events. By Bayes Rule and Total Probability:\\

\begin{align*}
	P(S_1|W) & = \frac{P(W|S_1)P(S_1)}{P(W|S_1)P(S_1)+P(W|S_2)P(S_2)} \\
			 & = \frac{e^{-\frac{3}{2}}0.6}{e^{-\frac{3}{2}}0.6+e^{-1}0.4} && \text{By the formula for $P(X)$ in an exponential distribution} \\
			 & = 0.47638386222 
\end{align*}


\end{document}

