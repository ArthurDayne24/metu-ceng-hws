\documentclass[12pt]{article}
\usepackage[utf8]{inputenc}
\usepackage{float}
\usepackage{amsmath}


\usepackage[hmargin=3cm,vmargin=6.0cm]{geometry}
%\topmargin=0cm
\topmargin=-2cm
\addtolength{\textheight}{6.5cm}
\addtolength{\textwidth}{2.0cm}
%\setlength{\leftmargin}{-5cm}
\setlength{\oddsidemargin}{0.0cm}
\setlength{\evensidemargin}{0.0cm}

\newcommand{\HRule}{\rule{\linewidth}{1mm}}

%misc libraries goes here
\usepackage{tikz}
\usetikzlibrary{automata,positioning}

\begin{document}

\noindent
\HRule \\[3mm]
\begin{flushright}
\LARGE \textbf{CENG 222}  \\[4mm]
\Large Statistical Methods for Computer Engineering \\[4mm]
\normalsize      Spring '2016-2017 \\
\Large   Assignment 4 \\
\normalsize Deadline: May 26, 23:59 \\
\normalsize Submission: via COW
\end{flushright}
\HRule

\section*{Student Information }
%Write your full name and id number between the colon and newline
%Put one empty space character after colon and before newline
Full Name :  Onur Can TIRTIR\\
Id Number :  2099380\\

% Write your answers below the section tags
\section*{Answer 9.8}
a)\\ \\
$\overline X = 42$ minutes\\
$\sigma = 5$ minutes\\
$1 - \alpha = 0.95$ then $\alpha = 0.05$
$\overline X \pm z_{\alpha / 2} \frac{\sigma}{\sqrt n}= \overline X \pm z_{0.025} \frac{5}{8} = 42 \pm q_{0.975} \frac{5}{8} = 42 \pm 1.225$\\
Then confidence interval is $[40.775, 43.225]$.\\

b)\\

\begin{align*}
P(40.775 < X < 43.225) &= P(\frac{40.775-40}{5} < Z < \frac{43.225-40}{5}) & \textit{Normalize } \\
                       &= P(0.151 < Z < 0.645) \\
                       &= P(z < 0.645) - P(z > 0.155) \\
                       &= 0.17895
\end{align*}

So the probability is $0.17895$.

\section*{Answer 9.16}
a)\\ \\
$\hat p_1 = 0.04$\\
$\hat p_2 = 0.06$\\
$1 - \alpha = 0.98$ then $\alpha = 0.02$\\
$n_1 = 250$ and $n_2 = 300$\\
$z_{alpha/2} = z_{0.01} = q_{0.99} = 2.3263$\\
We do not know standard deviation but we know $\hat p_1$ and $\hat p_2$.\\
Then answer is $\hat p_1 - \hat p_2 \pm z_{\alpha /2} \sqrt{\frac{(\hat p_1)(1-\hat p_1)}{n_1} + \frac{(\hat p_2)(1-\hat p_2)}{n_2}}\\ = -0.02 \pm 2.3263 \sqrt(\frac{0.06 * 0.94}{300} + \frac{0.04*0.96}{250})$\\

Then confidence interval is $[-0.62996, 0.22996] \approx [-0.63, 0.23]$ \\ \\ 

b)\\ \\
$\hat p_1 = 0.04$\\
$\hat p_2 = 0.06$\\

$z = \frac{\hat p_1 - \hat p_2}{\sqrt{\frac{\hat p_1(1 - \hat p_1)}{n_1} + \frac{\hat p_2(1 - \hat p_2)}{n_2}}} = \frac{-0.02}{\sqrt{\frac{0.06 * 0.94}{300} + \frac{0.04 * 0.96}{250}}} = -1.06$\\

Now, compute $p$ for test statistics\\
$p = 2P(Z < -1.06) = 0.28914$.\\

Since $p>0.02$, then we fail to reject our null hypothesis.\\
Then we can say that difference between the two lots' quality is not significant.

\section*{Answer 10.3}

We know $N = 100$\\
Doing some computations in Octave, we can say that\\
$\overline X = -0.058$\\
$\sigma = 1.058$\\

a)\\ \\
Now let us construct a frequency table four our data:\\
\begin{center}
    \begin{tabular}{| l | l |}
        \hline
        Interval & Observed \\
        \hline
        $< -2.0$ & 4 \\
        \hline
        $[-2.0, -1.5)$ & 4 \\
        \hline
        $[-1.5, -1.0)$ & 15 \\
        \hline 
        $[-1.0, -0.5)$ & 9 \\
        \hline
        $[-0.5, 0.0)$ & 22 \\
        \hline
        $[0.0, 0.5)$ & 15 \\
        \hline
        $[0.5, 1.0)$ & 12 \\
        \hline
        $[1.0, 1.5)$ & 11 \\
        \hline
        $[1.5, 2.0)$ & 7 \\
        \hline
        $\geq 2$ & 1 \\
        \hline
    \end{tabular}
\end{center}

Again by the help of Octave, using the expected normal distribution, calculate expected and observed frequencies and $\frac{(Observed\ -\ Expectation)^2}{Expectation}$ values and draw a new table.\\

\begin{center}
    \begin{tabular}{| l | l | l | l |} 
    \hline
        Interval       & Observed & Expectation & $(Obs-Exp)^2 / Exp$\\\hline
        $< -2.0$       & 4        & 3.32        & 0.14               \\\hline
        $[-2.0, -1.5)$ & 4        & 5.32        & 0.33               \\\hline
        $[-1.5, -1.0)$ & 15       & 10.02       & 2.48               \\\hline 
        $[-1.0, -0.5)$ & 9        & 15.14       & 2.49               \\\hline
        $[-0.5, 0.0)$  & 22       & 18.38       & 0.71               \\\hline
        $[0.0, 0.5)$   & 15       & 17.92       & 0.48               \\\hline
        $[0.5, 1.0)$   & 12       & 14.03       & 0.29               \\\hline
        $[1.0, 1.5)$   & 11       & 8.82        & 0.54               \\\hline
        $[1.5, 2.0)$   & 7        & 4.46        & 1.45               \\\hline
        $\geq 2$       & 1        & 2.59        & 0.97               \\\hline
    \end{tabular}
\end{center}

By summming up all the values in the last column, we get $X^2 = 9.883$ and degree of freedom $N - 1 = 9$ then $N = 10$. Compare $X^2$ value against the Chi-square distribution with given degree of freedom in Table A6 with a P-Value:\\
$P = P(X^2 > 9.883) =\ $ between 0.2 and 0.8, values of which are greater than the signicance level $0.05$.\\

Since we have no evidence that it does not belong to Normal Dist, then we can say that it may come from the normal distribution.

b)\\ \\
Given the distribution Uniform(-3, 3), $pdf(x) = 1 / (b-a)$ where $a \leq x \leq b$. Hence $a=-3$ and $b=3$. Now we should refill the table according to this distribution\\

\begin{center}
    \begin{tabular}{| l | l | l | l |} 
    \hline
        Interval       & Observed & Expectation & $(Obs-Exp)^2 / Exp$\\\hline
        $< -2.0$       & 4        & 16.67       & 9.63               \\\hline
        $[-2.0, -1.5)$ & 4        & 8.33        & 2.25               \\\hline
        $[-1.5, -1.0)$ & 15       & 8.33        & 5.33               \\\hline 
        $[-1.0, -0.5)$ & 9        & 8.33        & 0.05               \\\hline
        $[-0.5, 0.0)$  & 22       & 8.33        & 22.41              \\\hline
        $[0.0, 0.5)$   & 15       & 8.33        & 5.33               \\\hline
        $[0.5, 1.0)$   & 12       & 8.33        & 1.61               \\\hline
        $[1.0, 1.5)$   & 11       & 8.33        & 0.85               \\\hline
        $[1.5, 2.0)$   & 7        & 8.33        & 0.21               \\\hline
        $\geq 2$       & 1        & 16.67       & 14.73              \\\hline
    \end{tabular}
\end{center}

Calculate $X^2$ from the table by summing up the last column, get\\
$X^2 = 62.42$ and the degree of freedom is not changed, which is $N - 1 = 9\ then\ N = 10$.
Compare $X^2$ value against the Chi-square distribution with given degree of freedom in Table A6 with a P-Value:\\
$P = P(X^2 > 62.42) =\ $ p is less than 0.001, which is much and much less than our significance level $0.05$.\\
Then we can say that there is a strong evidence that the data cannot follow the Uniform(-3, 3) distribution.

c)\\
Theoretically yes. Because according to central limit theorem, if we have much and much larger data then it is possible that the data follows both Normal and Uniform distributions.





\end{document}
