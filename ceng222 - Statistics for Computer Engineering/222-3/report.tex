
\documentclass[12pt]{article}
\usepackage[utf8]{inputenc}
\usepackage{float}
\usepackage{amsmath}


\usepackage[hmargin=3cm,vmargin=6.0cm]{geometry}
%\topmargin=0cm
\topmargin=-2cm
\addtolength{\textheight}{6.5cm}
\addtolength{\textwidth}{2.0cm}
%\setlength{\leftmargin}{-5cm}
\setlength{\oddsidemargin}{0.0cm}
\setlength{\evensidemargin}{0.0cm}

\newcommand{\HRule}{\rule{\linewidth}{1mm}}

\begin{document}

\section*{Student Information } 
Full Name :  Onur Can TIRTIR\\
Id Number :  2099380\\

We first choose the number of Monte Carlo studies, $N$, so as to guarantee the condition given in the book. $N$ can be found by the formula:

$$ N \geq 0.25 (\dfrac{z_\alpha / 2}{\epsilon})^2$$

where $\alpha = 0.05$ and $\epsilon = 0.005$.\\

Here $N$ is 38415.\\

Then, for every Monte Carlo experiment, randomly get the number fish we catch according to \textit{poisson distribution} with $\lambda=3*4=12$. Then we follow the algorithm named as \textit{Rejection Method} to catch this number of fish. First choose the bounds for $X$, $a=0,\ b=3$, and $Y$, $c=0.7$ which is the maximum value for $f(x)$.

\begin{enumerate}
	\item Obtain standard uniform variables $U$ and $V$ from the function $rand$.
	\item Obtain $X = a + (b-a)U$ and $Y = c$.
	\item If $Y > f(X)$, reject the point $(X,Y)$ and return to \textit{step 2}, else accept the point. Accepting the point means that we succesfulyly catch the fish weight of $X$.
\end{enumerate}

When reached the number of fish to be caught, stop the algotrithm given above and save the $total\ weight$ of fish into the vector \textit{total weights}. then pass to next Monte Carlo study.

After finishing all the Monte Carlo studies, calculate \textit{a), b) and c)}. \\

\textbf{a)}\\
Count in how Monte Carlo studies we get more than $25\ kg$ of fish. First get a boolean vector, \textit{True if greater than 25, else False}, and get the ratio \textit{\#1s / total number of studies}.\\

Answer is approximately $0.215$.\\

\textbf{b)}\\
Take the mean of total weights to estimate the total weight.\\

Answer is approximately $20.15$.\\

\textbf{c)}\\
Calculate the standard deviation of all the Monte Carlo studies.\\

Answer is approximately $6.35$.

The less standard deviation, the less deviation from mean. The more deviation means that you have some or lots of outliers in your data, which reduces accuracy of the study.

\end{document}
